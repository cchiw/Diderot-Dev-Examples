% ---- ETD Document Class and Useful Packages ---- %
\documentclass{article}
\usepackage[margin=0.5in]{geometry}
\usepackage{subfigure,epsfig,amsfonts}
\usepackage{natbib}
\usepackage{amsmath}
\usepackage{amssymb}
\usepackage{amsthm}
%%%%%%%%%%%%%%%%%%%%%%%%%%%%%%%%%%%%
\usepackage{listings} % for code 
\usepackage{enumitem} % for creating lists
\usepackage{lineno, blindtext} % line numbers
\usepackage{xstring} %support if/else inside commands
\usepackage{longtable}
\usepackage{color}
\usepackage{multicol}
\usepackage{url}

\usepackage{mathtools}
\newtheorem{mydef}{Definition}

%%%%%%%%%%%%%%%%%%%%%%%%%%%%%%%%%%%%%%%
%%%%%%%%%%%%%%%%%%%%%%%%%%%%%%%%%%%%%%%
\newcommand{\doc}[0]{\textit{Doc}}
\newcommand{\branch}[0]{\textit{Diderot\_Dev}}
\newcommand{\exs}[0]{\textit{Exs}}

\newcommand{\details}[5]{
\textbf{Functionality}: {#1}\\
{Syntax}: {#2}\\
{Branch}: {#3}\\
{Text}:{#4}\\
{Issues}: {#5}\\
}

%%%%%%%%%%%%%%%%%%%%%%%%%%%%%%%%%%%%%%%
% Counters
%%%%%%%%%%%%%%%%%%%%%%%%%%%%%%%%%%%%%%%
\newcounter{bugs}
\newcounter{substInf}
\usepackage{enumerate}
\newcommand{\varMathFont}[1]{\text{\ttfamily{#1}}}
%%%%%%%%%%%%%%%%%%%%%%%%%%%%%%%%%%%%%%%
% Various Fonts 
%%%%%%%%%%%%%%%%%%%%%%%%%%%%%%%%%%%%%%%
\newcommand{\fontit}[1]{\textit{#1}}
\newcommand{\fontrelated}[1]{\textit{#1}}
\newcommand{\fontemph}[1]{\emph{#1}}
\newcommand{\fontR}{\mathbb{R}} % to do R


%%%%%%%%%%%%%%%%%%%%%%%%%%%%%%%%%%%%%%%
% EIN syntax
%%%%%%%%%%%%%%%%%%%%%%%%%%%%%%%%%%%%%%%
\newcommand{\name}{EIN}
\newcommand{\EinExp}[2]{\ensuremath{{\left\langle{#1}\right\rangle}_{#2}}}
\newcommand{\EinOp}[3]{\ensuremath{\lambda{}\,#1\EinExp{#2}{#3}}}


%%%%%%%%%%%%%%%%%%%%%%%%%%%%%%%%%%%%%%%
% EIN expression commands
%%%%%%%%%%%%%%%%%%%%%%%%%%%%%%%%%%%%%%%
\newcommand{\lift}[1]{\textbf{lift}_d({#1})}
\newcommand{\val}[1]{\textbf{val}({#1})} 
%ein differentiation
\newcommand{\einpart}[1]{\frac{\partial}{\partial x_{#1}}\diamond}
%indexing an ein expression
%ein exp 1 ...from 2 to 3
\newcommand{\expIndex}[3]{{#1}_{[{#2}/{#3}]}}
\newcommand{\expComp}[3]{{#1}\circ 
[\EinExp{#2}{\hat{#3}}]}

\newcommand{\expCompSingle}[3]{{#1}\circ 
e_c}
\newcommand{\expCompList}[3]{{#1}\circ [\EinExp{#2}{\hat{#3}},e_c ]  }

\newcommand{\expPolyWrap}[2]{\textit{PolyWrap}_{#1}({#2})}
\newcommand{\expPolyWrapD}[3]{\frac{\partial}{\partial x_{#3}} \textit{PolyWrap}_{#1}({#2})}

\newcommand{\expPolyTerm}[2]{{#1}^{#2}}
\newcommand{\expPolyTermD}[3]{
\frac{\partial}{\partial x_{#3}}{#1}^{#2}}

%%%%%%%%%%%%%%%%%%%%%%%%%%%%%%%%%%%%%%%
% Names of methods etc.
%%%%%%%%%%%%%%%%%%%%%%%%%%%%%%%%%%%%%%%
%\fontfamily{qcr}

\newcommand{\optshift}{\textit{Shift}}
\newcommand{\optEshape}{\textit{Eshape}}
\newcommand{\optshape}{\textit{Tshape}}
\newcommand{\optsplit}{\textit{Split}}
\newcommand{\optslice}{\textit{Slice}}
\newcommand{\defSlice}{OptSlice}
\newcommand{\checkname}[0]{\textit{DATm}}
\newcommand{\vname}[0]{\textit{DAVm}}
\newcommand{\lifted}{\textit{lifted}}
 
%%%%%%%%%%%%%%%%%%%%%%%%%%%%%%%%%%%%%%%
% Rewrites and Arrows
%%%%%%%%%%%%%%%%%%%%%%%%%%%%%%%%%%%%%%%
%rewrites on EIN operator level
\newcommand{\rewriteEINOP}{\xRightarrow[]{}}
\newcommand{\rewriteInitA}{\xRightarrow[init]{}}
\newcommand{\rewriteInitB}{\xRightarrow[subst]{surface}}
\newcommand{\rewriteInitC}{\xRightarrow[norm]{surface}}
\newcommand{\rewriteInitD}{\xRightarrow[split]{surface}}
\newcommand{\rewriteSubst}{\xRightarrow[subst]{}}
\newcommand{\rewriteSplit}{\xRightarrow[split]{}}
\newcommand{\rewriteSlice}{\xRightarrow[slice]{}}
\newcommand{\rewriteRecon}{\xRightarrow[recon]{}}


%rewrites on EIN expression level
\newcommand{\rewriteRule}{\xrightarrow[rule]{}}
\newcommand{\rewriteNorm}{\rewriteRule^*}
\newcommand{\rewriteShift}{\xrightarrow[shift]{}}
\newcommand{\rewriteEINEXP}{\xrightarrow[]{}}

%other arrows 
%ein-> math
\newcommand{\repSurface}{\xrightarrow[\text{Surface}]{EIN}}
% math -> to math
\newcommand{\rowSurface}{\longrightarrow_{direct-style}}
% ty -> ty in function declaration
\newcommand{\funcDec}{\rightarrow}
% tensor/field term -> ein expression in substitution 
\newcommand{\rowSubst}{\rightharpoonup}


%%%%%%%%%%%%%%%%%%%%%%%%%%%%%%%%%%%%%%%
% Method specific commands
%%%%%%%%%%%%%%%%%%%%%%%%%%%%%%%%%%%%%%%
\newcommand{\eshape}[2]{\optEshape{} ({#1})=\{{#2}\}}
\newcommand{\tshape}[1]{\optshape{}({#1})}
\newcommand{\ttshape}[2]{\optshape{}({#1})=\{{#2}\}}
\newcommand{\ttttshape}[4]{\optshape{}({#1},{#2},{#3})=\{{#4}\}}
  
 
%%%%%%%%%%%%%%%%%%%%%%%%%%%%%%%%%%%%%%%
% Types
%%%%%%%%%%%%%%%%%%%%%%%%%%%%%%%%%%%%%%%
% SSA-level types
\newcommand{\TensorTy}[1]{\mathbf{Ten}[#1]}
\newcommand{\FieldTy}[2]{\mathbf{Fld}(#1)[#2]}
\newcommand{\ImageTy}[2]{\mathbf{Img}(#1)[#2]}
\newcommand{\KernelTy}{\mathbf{Krn}}
% indexed Ein expression types
\newcommand{\genTTyB}{\ensuremath{\mathcal{T}}}      
\newcommand{\genFTyB}{\ensuremath{\mathcal{F}^{d}}}  % 
%indexed tensor and field type
\newcommand{\genTTyA}[1]{\ensuremath{(#1)\genTTyB{}}}      
\newcommand{\genFTyA}[1]{\ensuremath{(#1)\genFTyB{}}}  
%%%%%%%%%%%%%%%%%%%%
\newcommand{\genVTyA}[1]{\text{({$#1$})$\mathcal{I}_d$}}
\newcommand{\genHTyA}[1]{\text{({$#1$})$\mathcal{H}$}}
% mostly unchanged packages that are used in multiple papers
%various packages and definitions used in various documents
%
%
%%%%%%%%%%%%%%%%%%%%%%%%%%%%%%%%%%%%%%%%%%%%%%%%%%%%%%%
%%%%%%%% definitions %%%%%%%
\newtheorem{theorem}{Theorem}[section]
\newtheorem{lemma}[theorem]{Lemma}
%%\newtheorem{corollary}[theorem]{Corollary}
\newenvironment{definition}[1][Definition]{\begin{trivlist}
\item[\hskip \labelsep {\bfseries #1}]}{\end{trivlist}}
%\newenvironment{proof}[1][Proof]{\begin{trivlist}
%\item[\hskip \labelsep {\bfseries #1}]}{\end{trivlist}}
%%
%
\newcommand{\appref}[1]{Appendix~\ref{#1}}
\newcommand{\chapref}[1]{Chapter~\ref{#1}}
\newcommand{\secref}[1]{Section~\ref{#1}}
\newcommand{\tblref}[1]{Table~\ref{#1}}
\newcommand{\figref}[1]{Figure~\ref{#1}}
\newcommand{\eqqref}[1]{Equation~\ref{#1}}
\newcommand{\thmref}[1]{Theorem~\ref{#1}}
\newcommand{\lemref}[1]{Lemma~\ref{#1}}
%\newcommand{\listingref}[1]{Listing~\ref{#1}}
%\newcommand{\pref}[1]{{page~\pageref{#1}}}
\newcommand{\defref}[1]{Definition~\ref{#1} on page \pageref{#1}}
%\newcommand{\ruleref}[1]{Rule~\ref{#1}}
%%
\newcommand{\eg}{{\em e.g.}}
%\newcommand{\cf}{{\em cf.}}
\newcommand{\ie}{{\em i.e.}}
\newcommand{\etc}{{\em etc.\/}}
\newcommand{\naive}{na\"{\i}ve}
%\newcommand{\ala}{{\em \`{a} la\/}}
\newcommand{\etal}{{\em et al.\/}}
\newcommand{\role}{r\^{o}le}
%%\newcommand{\vs}{{\em vs.}}
%%\newcommand{\forte}{{fort\'{e}\/}}
%\newcommand{\Cplusplus}{\mbox{C\hspace{-.05em}
%\raisebox{.4ex}{\tiny\bf ++}}}
%%%%%%%% Colors %%%%%%%
%\definecolor{Red}{rgb}{0.9,0.0,0.0}  % fixme
%\definecolor{Green}{rgb}{0.0,0.4,0.0}
%\definecolor{Blue}{rgb}{0.0,0.0,0.9}
%\definecolor{DarkBlue}{rgb}{0.0,0.0,0.75}
%\definecolor{Midnight}{rgb}{0.0,0.0,0.5}
%\definecolor{Purple}{rgb}{0.5,0.0,0.4}
\definecolor{Black}{rgb}{0.0,0.0,0.0}
%\definecolor{Yellow}{rgb}{1.0,1.0, 0.25}
%\definecolor{Cyan}{rgb}{0.25,1.0, 1.0}
%
\newcommand{\cdColor}{Black}
%\newcommand{\kwColor}{DarkBlue}
%\newcommand{\comColor}{Red}
\newcommand{\mynote}[3]{\textcolor{#3}{\small\textbf{\textsf{{#1}: {#2}}}}}
\newcommand{\CC}[1]{\mynote{cc}{#1}{magenta}}
%
%
%
%%%%%%%% Some common math notation %%%%%%%
%%% double brackets
\newcommand{\LDB}{\ensuremath{[\mskip -3mu [}}
\newcommand{\RDB}{\ensuremath{]\mskip -3mu ]}}
%%
\newcommand{\dom}{\ensuremath{\mathrm{dom}}}
\newcommand{\rng}{\ensuremath{\mathrm{rng}}}
%%
%%% sets
\newcommand{\SET}[1]{\ensuremath{\{#1\}}}
\newcommand{\Fin}{\textrm{Fin}}     % finite power set
\newcommand{\DISJOINT}[2]{\ensuremath{#1 \pitchfork #2}}
\newcommand{\finsubset}{\mathrel{\stackrel{\textrm{fin}}{\subset}}}
%%
%%
%%% finite maps
\newcommand{\finmap}{\mathrel{\stackrel{\textrm{fin}}{\rightarrow}}}
%%\newcommand{\MAP}[2]{\SET{#1 \mapsto #2}}
%%\newcommand{\EXTEND}[2]{\ensuremath{#1{\pm}#2}}
%%\newcommand{\EXTENDone}[3]{\EXTEND{#1}{\MAP{#2}{#3}}}
%%\newcommand{\SUBST}[3]{\ensuremath{#1[#2\mapsto{}#3]}}
%%\newcommand{\SUBSTTWO}[5]{\ensuremath{#1[#2\mapsto{}#3,#4\mapsto{}#5]}}
%%
\definecolor{ngray}{rgb}{0.5,0.5,0.5}

\lstdefinelanguage{Diderot}{%
  % otherkeywords={|,\#},
  morekeywords={%
    bool,%
    die,%
    else,%
    false,field,foreach,%
identity,if,image,load,in,initially,input,int,%
    fem,
    kernel,%
    nan,new,%
    output,%
    real,%
    function,stabilize,strand,string,%
    inside,tensor,true,%
    update,%
    vec2,vec3,vec4,%
    zeros},
  sensitive,%
  morecomment=[s]{/*}{*/},%
  morecomment=[l]//,
  morestring=[b]"}%
  
  
\lstset{
  language=Diderot,
  %basicstyle= \ttfamily,%\rmfamily
  keywordstyle=\bfseries,
  commentstyle=\itshape,
  mathescape=true,
  %numbers=left,
  aboveskip=1.0em, belowskip=1.0em, lineskip=-0.10em,
   numberstyle=\color{ngray}
  }
 %generic input

%%%%%%%%%%%%%%%%%%%%%%%%%%%%%%%%%%%%

%% Use these commands to set biographic information for the title page:
\title{New language features}
\author{Charisee Chiw}


\begin{document}
\maketitle 
Our work has made it easier and faster to add to the Diderot language.
When we add an operator to the language we try to leverage our existing work.
This includes a concise representation in an expressive IR, the generic implementation of operators, and a robust testing model.
In the following, we illustrate the process of extending Diderot by providing examples.


\section{Defining operators between fields}
\subsection{Field Composition}

\begin{lstlisting}
field#k(d0)[$\sigma$] F0;
field#k(d1)[d0] F1;
field#k(d1)[$\sigma$] H = F0 $\circ$ F1; 
tensor[d1] pos;
tensor[$\sigma$] out = H(pos);
\end{lstlisting}

\paragraph{Representation}
We represent the field composition operator with the generic \name{} operator.
$$\rewriteInitA  H= \lambda(F,G) \EinExp{\expComp{F_\alpha}{G_\beta}{\beta}}{\hat{\alpha}}(F0,F1) \indent\text{ where } \hat{\alpha}=\sigma \text{ and } \hat{\beta} =[d0]$$
The field terms F and G represent fields in the composition.
F and G have separate index spaces. $F$ is bound by $\alpha$ and $G$ is bound by $\beta$.\\
 

\paragraph{Normalization}
%\paragraph{probe}
The probe of a composition
 $(\expComp{e_1}{e_2}{\beta})(x)$ 
 is rewritten depending on the structure of the outer term $e_1$. 
When the outer term is a constant the result does not depend on the composition operation.
$$(\expCompSingle{ c}{e_2}{\beta})(x) \rewriteRule{} c $$
Similarily, when the outer term is a non-field:
$$\begin{array}{lllllll}
(\expCompSingle{ \delta_{\alpha}}{e_2}{\beta})(x) &\rewriteRule{} \delta_{\alpha}& \indent &
(\expCompSingle{ \mathcal{E}_{\alpha}}{e_2}{\beta})(x) &\rewriteRule{} \mathcal{E}_{\alpha}\\
(\expCompSingle{ Z_{\alpha}}{e_2}{\beta})(x) &\rewriteRule{} Z_{\alpha} &&
(\expCompSingle{ \lift{e}}{e_2}{\beta})(x) &\rewriteRule{} e &
\end{array}$$
The probe operator is pushed past arithmetic operators:
$$\begin{array}{lllll}
(\expCompSingle{\odot_1 e}{e_2}{\beta})(x) &\rewriteRule{} \odot_1 (  \expCompSingle{e}{e_2}{\beta})(x) \\
(\expCompSingle{\sum_{\hat{\alpha}} e}{e_2}{\beta})(x) &\rewriteRule{} \sum_{\hat{\alpha}} (  \expCompSingle{e}{e_2}{\beta})(x) \\
\end{array}$$
The probe is distributed: 
$$\begin{array}{lllll}
(\expCompSingle{(e_a - e_b)}{e_2}{\beta})(x) &\rewriteRule{} (\expCompSingle{e_a}{e_2}{\beta})(x) -(\expCompSingle{e_b}{e_2}{\beta})(x) \\
(\expCompSingle{(e_a * e_b * e_s )}{e_2}{\beta})(x) &\rewriteRule{} (\expCompSingle{e_a}{e_2}{\beta})(x) * (\expCompSingle{e_b}{e_2}{\beta})(x)*(\expCompSingle{e_s}{e_2}{\beta})(x) \end{array}$$



%\paragraph{derivative}
The derivative of a field composition is applied by using the chain rule.\\
$$\nabla (F \circ G) \rowSurface (\nabla F \circ G)\bullet (\nabla G) $$
The derivative of a field composition of two fields is represented in the \name{} IR as 
$$\nabla_j (\expComp{F_\alpha}{G_{i\beta}}{i\beta}) \rewriteRule{}
\sum_{\hat{k}}( (\expComp{\nabla_k  F_\alpha}{G_{i\beta}}{i\beta}) * (\nabla_j G_{k\beta}) )$$
Generally we use the rewrite rule to apply the rewrite between two \name{} expressions:\\
$$\nabla_j (\expComp{e_1}{e_2}{i\beta}) \rewriteRule{}
\sum_{\hat{k}}( (\expComp{\nabla_k  e_1}{e_2}{i\beta}) * (\nabla_j \expIndex{e_2}{i}{k}) )$$


%\paragraph{flatten}
Flatten composition operator
$$ \expCompSingle{ (\expComp{a}{b}{m})}{c}{n} 
\rewriteRule{}
\expCompList{a}{b}{m}
$$

$$ \expComp{a}{\expCompSingle{b}{c}{n}}{m} 
\rewriteRule{}
\expCompList{a}{b}{m}
$$



\paragraph{Split}
After being normalized the probed composition operator is split into several probes.
$$
\text{\lstinline!out!} = \lambda F,G,x \EinExp{\expComp{F_\alpha}{G_\beta}{\beta}(x)}{\alpha}(F0, F1, x)
\rewriteSplit{}
\begin{array}{rl}
\lstinline!t!_0 = & \lambda G,x\EinExp{G_\beta(x)}{\hat{\beta}}(F1,x)\\
\text{\lstinline!out!}= & \lambda F,x \EinExp{F_\alpha(x)}{\alpha}(F0, \lstinline!t!_0)
 \end{array}$$
 
  \subsection{Concat}
A user can define new tensors by concatenating tensors together.
A Diderot program
\begin{lstlisting}[mathescape=true]
  tensor[d$_1$]S; 
  tensor[d$_2$]T;
  tensor[d$_1$,d$_2$]M = [S,T]; 
\end{lstlisting}

A user can refer to components of tensor fields by using the slice operation as shown in the following code.
A Diderot program
\begin{lstlisting}[mathescape=true]
  field#k(d)[d$_1$,d$_2$]A; 
  field#k(d)[d$_1$,d$_2$]B;
  field#k(d)[d$_1$]F = A[:,0]; 
  field#k(d)[d$_1$]G = B[:,1];
\end{lstlisting}

We would like to provide a way to define new tensors fields by concatenating components together.
Using the tensor field variables \lstinline!F! and  \lstinline!G! defined earlier in the program the Diderot code should support the line 
\begin{lstlisting}[mathescape=true]
  field#k(d)[d$_1$,d$_1$]H = [F, G];
\end{lstlisting}

\noindent We illustrate the structure of \lstinline[mathescape=true]!H! below.  
\begin{displaymath}
  H=\left[ \begin{array}{ll}
  F_0  & F_1\\
  G_0  &G_1
  \end{array}
 \right]  
 \end{displaymath}
 
\paragraph{Representation}
We can use \name{} expressions as building blocks to represent field concatenation. 
In \name{} each field term is represented by an  expression and it is enabled with a delta function 
$$\rewriteInitA{} H=\lambda F,G.\EinExp{F_{j }\delta_{0i}+G_{j }\delta_{1i}}{i:2,j:2}\text{(\lstinline[mathescape=true]!F,G!)}$$
After substitution the new \name{} operator would be 
$$\rewriteSubst{}H=\lambda A,B.\EinExp{A_{j 0}\delta_{0i}+B_{j 1}\delta_{1i}}{\hat{i}\hat{j}}\text{(\lstinline[mathescape=true]!A,B!)}.$$
In the compiler we choose to create generic versions of an \name{} operator that can be instantiated to certain types.
$$\begin{array}{l}
\lambda F,G. \EinExp{F_{\alpha }\delta_{0i}+G_{\alpha }\delta_{1i}}{{i:2} \hat{\alpha} }\text{(\lstinline[mathescape=true]!F,G!)}\\
\lambda F,G, H.\EinExp{F_{\alpha }\delta_{0i}+G_{\alpha }\delta_{1i}+H_{\alpha }\delta_{2i}}{{i:3} \hat{\alpha} }\text{(\lstinline[mathescape=true]!F,G,H!)}
\end{array}$$
To implement this operator we  need to add to cases to the Diderot typechecker and add the generic representations but not much else.
Since we are solely using existing \name{} expressions to represent this computation, we can rely on the existing code to handle the \name{} syntax.


\subsection{Abs, Max, and Min}


Implementation lift max  and min operators to the field level and represents the magnitude of scalar fields with absolute expressions. 
Extra syntax is created to support differentiation.


\paragraph{Design}
Need to add new syntax to support differentiation.
$$\begin{array}{llll}
cond &=& e>e \mid e<e & \text{conditional }\\
e&=&  \text{Max}(a,b)  \mid \text{Min}(a,b) & \text{Binary \name{} operators}\\
&\mid & \text{if}(cond, e, e)  &\text{If wrapper returns tensor-valued expression}\\
&\mid &\text{Abs}(e) &\text{Absolute function}\\
&\mid &\text{Sgn} (e)& \text{Returns Sign (-1, 0, 1)}\\
\end{array}$$

 \paragraph{Differentiation rules}
 Differentiation of an absolute expression:
  $$\frac{\partial}{\partial x_\alpha} \text{abs}(e) \rewriteEINEXP
( \frac{\partial}{\partial x_\alpha} e) *(\text{Sgn} (e))
 $$ 

Differentiation creates an if wrapper expressions.
 $$\frac{\partial}{\partial x_\alpha} \text{Max}(a,b) \rewriteEINEXP if(a>b, \frac{\partial}{\partial x_\alpha}a, \frac{\partial}{\partial x_\alpha}b)  $$
 $$\frac{\partial}{\partial x_\alpha}\text{Min}(a,b) \rewriteEINEXP if(a<b, \frac{\partial}{\partial x_\alpha}  a, \frac{\partial}{\partial x_\alpha} b)  $$
  Differentiation of an If wrapper is pushed to leaves.
   $$\frac{\partial}{\partial x_\alpha} \text{If}(cond, c,d) \rewriteEINEXP if(cond, \frac{\partial}{\partial x_\alpha}c, \frac{\partial}{\partial x_\alpha}d)  $$
  \paragraph{other  rules}
Otherwise, Max and Min are treated like  other binary operators. The following pushes the probes to the leaves.
  $$ (\text{Max}(a,b))(x) \rewriteEINEXP \text{Max}(a(x),  b(x)) $$

   
   
\section{New ways to define a field}
\subsection{Polynomial}
It is natural to define a field with a polynomial expression. $$F = x^3$$
In the surface language we added function \lstinline[mathescape=true]!poly()!.
The first argument is a variable and the second argument is a field definition.
\begin{lstlisting}[mathescape=true]
vec2 x;
ofield#2(2)[2] V = poly(x, x);
ofield#2(2)[] S = poly(x, x$^2$);
\end{lstlisting}
This allows the programmer to differentiate this type of field.
\begin{lstlisting}[mathescape=true]
ofield#1(2)[2] GS = $\nabla$S;
\end{lstlisting}
We illustrate the expected structure below:
\begin{displaymath}
  V=\left[ \begin{array}{ll}
  V_0 \\  V_1  \\
  \end{array}
 \right]   \indent\indent 
  \nabla \otimes V=\left[ \begin{array}{ll}
 1&   0 \\  0  &  1  \\
  \end{array}
 \right]   \indent\indent 
   S=  V_0 ^2+  V_1^2  \\
  \indent\indent
    \nabla S=\left[ \begin{array}{ll}
  2*V_0 \\  2*V_1  \\
  \end{array}
 \right] 
 \end{displaymath}

\paragraph{Representation}.\\
As an ongoing example:
A field F is defined by taking the cube of the input variable.\begin{lstlisting}[mathescape=true]
real p; 
ofield#1(2)[] F = poly(p, $p^3$);
tensor[] out = F(pos);
\end{lstlisting}

$\rewriteInitA \begin{array}{ll} \lstinline[mathescape=true]!F!=\lambda ()\EinExp{\expPolyWrap{p}{{p}^3}}{}()\\
\lstinline[mathescape=true]!out!=\lambda(F,x) \EinExp{F(x)}{}\lstinline[mathescape=true]!(F,x)!\end{array}$\\
\\
Substitution creates:\\
$\rewriteSubst{}$ \lstinline[mathescape=true]!out!=$\lambda(x) \EinExp{\expPolyWrap{p}{p^3}(x)}{}$
\lstinline[mathescape=true]!(x)!
\paragraph{Replace polynomial variable}.\\
The variable $p$ represents a vector of length 2, where p= [X,Y].\\
The term $P_0$ represents the 0th component of the vector, or X.
In the following, we will use the terms $X$ and $Y$, in place of $P_0$ and $P_1$.


The polynomial variable is instantiated with the position.\\
$\rewriteEINOP $ \lstinline[mathescape=true]!out!=$\lambda(p) \EinExp{e}{}$\lstinline[mathescape=true]!(x)! where e=$p*p*p$.\\
The \name{} term ($p$) is replaced with an \name{} term that represents the vector components.
 In the 2-d case there are two terms indexed with a constant index in 
$p = X \delta_{0i}+Y \delta_{1i}$\\


Occurrances for P are replaced inside the expression:\\ 
$\rewriteEINEXP (P_0 \delta_{0i}+P_1 \delta_{1i})*(P_0 \delta_{0i}+P_1 \delta_{1i})*(P_0 \delta_{0i}+P_1 \delta_{1i})$.\\
$= (X \delta_{0i}+Y \delta_{1i})*(X \delta_{0i}+Y \delta_{1i})*(X \delta_{0i}+Y\delta_{1i})$.\\




\paragraph{Normalization}.\\
Similar terms are collected:\\
$$P_0*P_0\rewriteEINEXP \expPolyTerm{P_0}{2} \indent \text{or}\indent X*X\rewriteEINEXP \expPolyTerm{X}{2} $$ 
The differentiation operator is distributed over the \name{} term, as usual,  and pushed to a polynomial term\\
 $$\frac{\partial}{\partial x_i} (\expPolyTerm{P_0}{2}+e) \rewriteEINEXP \expPolyTermD{P_0}{2}{i} + \frac{\partial}{\partial x_i}  e
 \indent \text{or}\indent
 \frac{\partial}{\partial x_i} (\expPolyTerm{X}{2}+e) \rewriteEINEXP \expPolyTermD{X}{2}{i} +\frac{\partial}{\partial x_i}  e
 $$


\paragraph{Evaluation}.\\
During the evaluation the variable index in a differentiation operator is bound to a number.
An \name{} term such as $\expPolyTermD{P_c}{n}{i}$ is evaluated.\\
When $i$ and $c$ are both 0:\\ 
$$\frac{\partial}{\partial x_0} X \rewriteEINEXP  1 
 \indent  \frac{\partial}{\partial x_0} X^2 \rewriteEINEXP  2*\expPolyTerm{X}{}  
  \indent \frac{\partial}{\partial x_0} X^3  \rewriteEINEXP  3*X^{2} $$ 
When $i$ and $c$ are both 1:\\ 
$$
\expPolyTermD{Y}{}{1} \rewriteEINEXP  1 
 \indent  \expPolyTermD{Y}{2}{1} \rewriteEINEXP  2*\expPolyTerm{Y}{}  
  \indent  \expPolyTermD{Y}{3}{1} \rewriteEINEXP  3*\expPolyTerm{Y}{2}  
   $$  
When $i$ and $c$ are not the same\\ 
$$ \expPolyTermD{X}{n}{1} \rewriteEINEXP  0 
\indent \expPolyTermD{Y}{n}{1} \rewriteEINEXP  0  $$ 


 \end{document}

