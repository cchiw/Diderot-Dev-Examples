
\begin{description}[noitemsep]
\item [Lingo] Files and Variables
\begin{itemize}[noitemsep]
 \item \textit{Frame} is the file frame.py
 \item  A  \textit{label} is the unique identifier for each test/group of tests. It is printed on the command line or in the resulting text file.
 For instance, label ``p$\_$o27$\_$o0...." . The numbers 27 and 0 refer to unique identifiers for operators.
 \end{itemize}
\item [Getting Started] short instructions
\begin{enumerate}
\item Checkout github directory for \datm{}
\item Change cpath in \textit{Frame} to your absolute path to diderot branches. See \textbf{Set up} about other variables you might want to change.
\item Run command ``python3 cte.py 1 0". See Section \textbf{Running DATm}.
\end{enumerate}
\item [Set up] Variables, Testing frame settings
\begin{itemize}[noitemsep]
\item Q:Can you test different branches? \\ 
A:Yes, comment in the right  ``s$\_$branch" variable in \textit{Frame}  \\
\tab or add a new branch name in "branch$\_$other" in "shared/base$\_$constants.py"
\item Q:Can I do an exhaustive search for test cases? \\
A: Yes, set s$\_$random$\_$range in \textit{Frame}  to 0
\item Q:Can I do an random search for test cases? \\
A: Yes, set s$\_$random$\_$range in \textit{Frame} to a number x where the probability of a single test case being generated is 1 in x+1
\item Q:What else can I change about the testing environment? \\
A: You can comment in and out variables in \textit{Frame}. This includes variables to change the coefficient order, number of samples, number of operators, type of arguments,.... See Pg 102 in \diss{}
\end{itemize}
\item [Running DATm] command-line commands and scope
\begin{itemize}[noitemsep]
\item Q:How to run everything \\ 
A:python3 cte.py 0
\item Q:How to run a tests for a single operator \\
A: python3  cte.py 1 id, where id is a number
\item Q: Can we target tests?\\
A: Yes we can use the label to inspire command line arguments.
For instance, the test case(s) with\\  label ``p$\_$o27$\_$o0$\_$t0$\_$tN$\_$tN$\_$$\_$l2"
 can rerun with command python cte.py 3  27 0 0, b   where ``3" is the number of arguments and ``27 0 0" refer to the integers in the label before ``tN"
 \end{itemize}
\item [Results] passes/fails
\begin{itemize}[noitemsep]
\item Q: Great, everything is running now, but how do I look at the results?\\
A: In the directory rst/stash are several text files that record the test cases.\\ \tab
results$\_$final.txt:Testing frame and the results of each test case \\ \tab
results$\_$terrible.txt: Reports test cases with errors\\ \tab
results$\_$ty.txt:Test labels and Types 
\item Q: Can we rerun tests (group of tests) that failed?\\
A: Yes, see question on targeted testing
\item Q: Has it been useful at finding bugs?\\
A: Yes, See \paperAst{}
\end{itemize}
\item [Development] command-line commands and scope
\begin{itemize}[noitemsep]
\item Q:How do I add a new operator? A:See Pg 113 in \diss{}.
\begin{enumerate}
\item Add to operator constant. shared/obj\_operator.py. 
\item Add case to type-checker. shared/obj\_typechecker.py. 
\item Add way to evaluate that operator applied to polynomials in nc/nc\_eval.py.
\end{enumerate}
\item Q: What is the code organization?\\
A: :) 
\end{itemize}
\end{description}