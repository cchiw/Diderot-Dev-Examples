
  \subsection{Concat}
A user can define new tensors by concatenating tensors together.
A Diderot program
\begin{lstlisting}[mathescape=true]
  tensor[d$_1$]S; 
  tensor[d$_2$]T;
  tensor[d$_1$,d$_2$]M = [S,T]; 
\end{lstlisting}

A user can refer to components of tensor fields by using the slice operation as shown in the following code.
A Diderot program
\begin{lstlisting}[mathescape=true]
  field#k(d)[d$_1$,d$_2$]A; 
  field#k(d)[d$_1$,d$_2$]B;
  field#k(d)[d$_1$]F = A[:,0]; 
  field#k(d)[d$_1$]G = B[:,1];
\end{lstlisting}

We would like to provide a way to define new tensors fields by concatenating components together.
Using the tensor field variables \lstinline!F! and  \lstinline!G! defined earlier in the program the Diderot code should support the line 
\begin{lstlisting}[mathescape=true]
  field#k(d)[d$_1$,d$_1$]H = [F, G];
\end{lstlisting}

\noindent We illustrate the structure of \lstinline[mathescape=true]!H! below.  
\begin{displaymath}
  H=\left[ \begin{array}{ll}
  F_0  & F_1\\
  G_0  &G_1
  \end{array}
 \right]  
 \end{displaymath}
 
\paragraph{Representation}
We can use \name{} expressions as building blocks to represent field concatenation. 
In \name{} each field term is represented by an  expression and it is enabled with a delta function 
$$\rewriteInitA{} H=\lambda F,G.\EinExp{F_{j }\delta_{0i}+G_{j }\delta_{1i}}{i:2,j:2}\text{(\lstinline[mathescape=true]!F,G!)}$$
After substitution the new \name{} operator would be 
$$\rewriteSubst{}H=\lambda A,B.\EinExp{A_{j 0}\delta_{0i}+B_{j 1}\delta_{1i}}{\hat{i}\hat{j}}\text{(\lstinline[mathescape=true]!A,B!)}.$$
In the compiler we choose to create generic versions of an \name{} operator that can be instantiated to certain types.
$$\begin{array}{l}
\lambda F,G. \EinExp{F_{\alpha }\delta_{0i}+G_{\alpha }\delta_{1i}}{{i:2} \hat{\alpha} }\text{(\lstinline[mathescape=true]!F,G!)}\\
\lambda F,G, H.\EinExp{F_{\alpha }\delta_{0i}+G_{\alpha }\delta_{1i}+H_{\alpha }\delta_{2i}}{{i:3} \hat{\alpha} }\text{(\lstinline[mathescape=true]!F,G,H!)}
\end{array}$$
To implement this operator we  need to add to cases to the Diderot typechecker and add the generic representations but not much else.
Since we are solely using existing \name{} expressions to represent this computation, we can rely on the existing code to handle the \name{} syntax.

