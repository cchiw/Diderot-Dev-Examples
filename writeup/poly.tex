\subsection{Defining fields with polynomials}
It is natural to define a field with a polynomial expression. $$F = x^3$$
In the surface language we added function \lstinline[mathescape=true]!poly()!.
The first argument is a variable and the second argument is a field definition.
\begin{lstlisting}[mathescape=true]
vec2 x;
ofield#2(2)[2] V = poly(x, x);
ofield#2(2)[] S = poly(x, x$^2$);
\end{lstlisting}
This allows the programmer to differentiate this type of field.
\begin{lstlisting}[mathescape=true]
ofield#1(2)[2] GS = $\nabla$S;
\end{lstlisting}
We illustrate the expected structure below:
\begin{displaymath}
  V=\left[ \begin{array}{ll}
  V_0 \\  V_1  \\
  \end{array}
 \right]   \indent\indent 
  \nabla \otimes V=\left[ \begin{array}{ll}
 1&   0 \\  0  &  1  \\
  \end{array}
 \right]   \indent\indent 
   S=  V_0 ^2+  V_1^2  \\
  \indent\indent
    \nabla S=\left[ \begin{array}{ll}
  2*V_0 \\  2*V_1  \\
  \end{array}
 \right] 
 \end{displaymath}

\paragraph{Representation}.\\
As an ongoing example:
A field F is defined by taking the cube of the input variable.\begin{lstlisting}[mathescape=true]
real p; 
ofield#1(2)[] F = poly(p, $p^3$);
tensor[] out = F(pos);
\end{lstlisting}

$\rewriteInitA \begin{array}{ll} \lstinline[mathescape=true]!F!=\lambda ()\EinExp{\expPolyWrap{p}{{p}^3}}{}()\\
\lstinline[mathescape=true]!out!=\lambda(F,x) \EinExp{F(x)}{}\lstinline[mathescape=true]!(F,x)!\end{array}$\\
\\
Substitution creates:\\
$\rewriteSubst{}$ \lstinline[mathescape=true]!out!=$\lambda(x) \EinExp{\expPolyWrap{p}{p^3}(x)}{}$
\lstinline[mathescape=true]!(x)!
\paragraph{Replace polynomial variable}.\\
The variable $p$ represents a vector of length 2, where p= [X,Y].\\
The term $P_0$ represents the 0th component of the vector, or X.
In the following, we will use the terms $X$ and $Y$, in place of $P_0$ and $P_1$.


The polynomial variable is instantiated with the position.\\
$\rewriteEINOP $ \lstinline[mathescape=true]!out!=$\lambda(p) \EinExp{e}{}$\lstinline[mathescape=true]!(x)! where e=$p*p*p$.\\
The \name{} term ($p$) is replaced with an \name{} term that represents the vector components.
 In the 2-d case there are two terms indexed with a constant index in 
$p = X \delta_{0i}+Y \delta_{1i}$\\


Occurrances for P are replaced inside the expression:\\ 
$\rewriteEINEXP (P_0 \delta_{0i}+P_1 \delta_{1i})*(P_0 \delta_{0i}+P_1 \delta_{1i})*(P_0 \delta_{0i}+P_1 \delta_{1i})$.\\
$= (X \delta_{0i}+Y \delta_{1i})*(X \delta_{0i}+Y \delta_{1i})*(X \delta_{0i}+Y\delta_{1i})$.\\




\paragraph{Normalization}.\\
Similar terms are collected:\\
$$P_0*P_0\rewriteEINEXP \expPolyTerm{P_0}{2} \indent \text{or}\indent X*X\rewriteEINEXP \expPolyTerm{X}{2} $$ 
The differentiation operator is distributed over the \name{} term, as usual,  and pushed to a polynomial term\\
 $$\frac{\partial}{\partial x_i} (\expPolyTerm{P_0}{2}+e) \rewriteEINEXP \expPolyTermD{P_0}{2}{i} + \frac{\partial}{\partial x_i}  e
 \indent \text{or}\indent
 \frac{\partial}{\partial x_i} (\expPolyTerm{X}{2}+e) \rewriteEINEXP \expPolyTermD{X}{2}{i} +\frac{\partial}{\partial x_i}  e
 $$


\paragraph{Evaluation}.\\
During the evaluation the variable index in a differentiation operator is bound to a number.
An \name{} term such as $\expPolyTermD{P_c}{n}{i}$ is evaluated.\\
When $i$ and $c$ are both 0:\\ 
$$\frac{\partial}{\partial x_0} X \rewriteEINEXP  1 
 \indent  \frac{\partial}{\partial x_0} X^2 \rewriteEINEXP  2*\expPolyTerm{X}{}  
  \indent \frac{\partial}{\partial x_0} X^3  \rewriteEINEXP  3*X^{2} $$ 
When $i$ and $c$ are both 1:\\ 
$$
\expPolyTermD{Y}{}{1} \rewriteEINEXP  1 
 \indent  \expPolyTermD{Y}{2}{1} \rewriteEINEXP  2*\expPolyTerm{Y}{}  
  \indent  \expPolyTermD{Y}{3}{1} \rewriteEINEXP  3*\expPolyTerm{Y}{2}  
   $$  
When $i$ and $c$ are not the same\\ 
$$ \expPolyTermD{X}{n}{1} \rewriteEINEXP  0 
\indent \expPolyTermD{Y}{n}{1} \rewriteEINEXP  0  $$ 
