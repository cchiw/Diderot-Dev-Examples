% ---- ETD Document Class and Useful Packages ---- %
\documentclass{article}
\usepackage[margin=0.5in]{geometry}
\usepackage{subfigure,epsfig,amsfonts}
\usepackage{natbib}
\usepackage{amsmath}
\usepackage{amssymb}
\usepackage{amsthm}
%%%%%%%%%%%%%%%%%%%%%%%%%%%%%%%%%%%%
\usepackage{listings} % for code 
\usepackage{enumitem} % for creating lists
\usepackage{lineno, blindtext} % line numbers
\usepackage{xstring} %support if/else inside commands
\usepackage{longtable}
\usepackage{color}
\usepackage{multicol}
\usepackage{url}

\usepackage{mathtools}
\newtheorem{mydef}{Definition}

%%%%%%%%%%%%%%%%%%%%%%%%%%%%%%%%%%%%%%%
%%%%%%%%%%%%%%%%%%%%%%%%%%%%%%%%%%%%%%%
\newcommand{\doc}[0]{\textit{Doc}}
\newcommand{\branch}[0]{\textit{Diderot\_Dev}}
\newcommand{\exs}[0]{\textit{Exs}}

\newcommand{\details}[5]{
\textbf{Functionality}: {#1}\\
{Syntax}: {#2}\\
{Branch}: {#3}\\
{Text}:{#4}\\
{Issues}: {#5}\\
}

%%%%%%%%%%%%%%%%%%%%%%%%%%%%%%%%%%%%%%%
% Counters
%%%%%%%%%%%%%%%%%%%%%%%%%%%%%%%%%%%%%%%
\newcounter{bugs}
\newcounter{substInf}
\usepackage{enumerate}
\newcommand{\varMathFont}[1]{\text{\ttfamily{#1}}}
%%%%%%%%%%%%%%%%%%%%%%%%%%%%%%%%%%%%%%%
% Various Fonts 
%%%%%%%%%%%%%%%%%%%%%%%%%%%%%%%%%%%%%%%
\newcommand{\fontit}[1]{\textit{#1}}
\newcommand{\fontrelated}[1]{\textit{#1}}
\newcommand{\fontemph}[1]{\emph{#1}}
\newcommand{\fontR}{\mathbb{R}} % to do R


%%%%%%%%%%%%%%%%%%%%%%%%%%%%%%%%%%%%%%%
% EIN syntax
%%%%%%%%%%%%%%%%%%%%%%%%%%%%%%%%%%%%%%%
\newcommand{\name}{EIN}
\newcommand{\EinExp}[2]{\ensuremath{{\left\langle{#1}\right\rangle}_{#2}}}
\newcommand{\EinOp}[3]{\ensuremath{\lambda{}\,#1\EinExp{#2}{#3}}}


%%%%%%%%%%%%%%%%%%%%%%%%%%%%%%%%%%%%%%%
% EIN expression commands
%%%%%%%%%%%%%%%%%%%%%%%%%%%%%%%%%%%%%%%
\newcommand{\lift}[1]{\textbf{lift}_d({#1})}
\newcommand{\val}[1]{\textbf{val}({#1})} 
%ein differentiation
\newcommand{\einpart}[1]{\frac{\partial}{\partial x_{#1}}\diamond}
%indexing an ein expression
%ein exp 1 ...from 2 to 3
\newcommand{\expIndex}[3]{{#1}_{[{#2}/{#3}]}}
\newcommand{\expComp}[3]{{#1}\circ 
[\EinExp{#2}{\hat{#3}}]}

\newcommand{\expCompSingle}[3]{{#1}\circ 
e_c}
\newcommand{\expCompList}[3]{{#1}\circ [\EinExp{#2}{\hat{#3}},e_c ]  }

\newcommand{\expPolyWrap}[2]{\textit{PolyWrap}_{#1}({#2})}
\newcommand{\expPolyWrapD}[3]{\frac{\partial}{\partial x_{#3}} \textit{PolyWrap}_{#1}({#2})}

\newcommand{\expPolyTerm}[2]{{#1}^{#2}}
\newcommand{\expPolyTermD}[3]{
\frac{\partial}{\partial x_{#3}}{#1}^{#2}}

%%%%%%%%%%%%%%%%%%%%%%%%%%%%%%%%%%%%%%%
% Names of methods etc.
%%%%%%%%%%%%%%%%%%%%%%%%%%%%%%%%%%%%%%%
%\fontfamily{qcr}

\newcommand{\optshift}{\textit{Shift}}
\newcommand{\optEshape}{\textit{Eshape}}
\newcommand{\optshape}{\textit{Tshape}}
\newcommand{\optsplit}{\textit{Split}}
\newcommand{\optslice}{\textit{Slice}}
\newcommand{\defSlice}{OptSlice}
\newcommand{\checkname}[0]{\textit{DATm}}
\newcommand{\vname}[0]{\textit{DAVm}}
\newcommand{\lifted}{\textit{lifted}}
 
%%%%%%%%%%%%%%%%%%%%%%%%%%%%%%%%%%%%%%%
% Rewrites and Arrows
%%%%%%%%%%%%%%%%%%%%%%%%%%%%%%%%%%%%%%%
%rewrites on EIN operator level
\newcommand{\rewriteEINOP}{\xRightarrow[]{}}
\newcommand{\rewriteInitA}{\xRightarrow[init]{}}
\newcommand{\rewriteInitB}{\xRightarrow[subst]{surface}}
\newcommand{\rewriteInitC}{\xRightarrow[norm]{surface}}
\newcommand{\rewriteInitD}{\xRightarrow[split]{surface}}
\newcommand{\rewriteSubst}{\xRightarrow[subst]{}}
\newcommand{\rewriteSplit}{\xRightarrow[split]{}}
\newcommand{\rewriteSlice}{\xRightarrow[slice]{}}
\newcommand{\rewriteRecon}{\xRightarrow[recon]{}}


%rewrites on EIN expression level
\newcommand{\rewriteRule}{\xrightarrow[rule]{}}
\newcommand{\rewriteNorm}{\rewriteRule^*}
\newcommand{\rewriteShift}{\xrightarrow[shift]{}}
\newcommand{\rewriteEINEXP}{\xrightarrow[]{}}

%other arrows 
%ein-> math
\newcommand{\repSurface}{\xrightarrow[\text{Surface}]{EIN}}
% math -> to math
\newcommand{\rowSurface}{\longrightarrow_{direct-style}}
% ty -> ty in function declaration
\newcommand{\funcDec}{\rightarrow}
% tensor/field term -> ein expression in substitution 
\newcommand{\rowSubst}{\rightharpoonup}


%%%%%%%%%%%%%%%%%%%%%%%%%%%%%%%%%%%%%%%
% Method specific commands
%%%%%%%%%%%%%%%%%%%%%%%%%%%%%%%%%%%%%%%
\newcommand{\eshape}[2]{\optEshape{} ({#1})=\{{#2}\}}
\newcommand{\tshape}[1]{\optshape{}({#1})}
\newcommand{\ttshape}[2]{\optshape{}({#1})=\{{#2}\}}
\newcommand{\ttttshape}[4]{\optshape{}({#1},{#2},{#3})=\{{#4}\}}
  
 
%%%%%%%%%%%%%%%%%%%%%%%%%%%%%%%%%%%%%%%
% Types
%%%%%%%%%%%%%%%%%%%%%%%%%%%%%%%%%%%%%%%
% SSA-level types
\newcommand{\TensorTy}[1]{\mathbf{Ten}[#1]}
\newcommand{\FieldTy}[2]{\mathbf{Fld}(#1)[#2]}
\newcommand{\ImageTy}[2]{\mathbf{Img}(#1)[#2]}
\newcommand{\KernelTy}{\mathbf{Krn}}
% indexed Ein expression types
\newcommand{\genTTyB}{\ensuremath{\mathcal{T}}}      
\newcommand{\genFTyB}{\ensuremath{\mathcal{F}^{d}}}  % 
%indexed tensor and field type
\newcommand{\genTTyA}[1]{\ensuremath{(#1)\genTTyB{}}}      
\newcommand{\genFTyA}[1]{\ensuremath{(#1)\genFTyB{}}}  
%%%%%%%%%%%%%%%%%%%%
\newcommand{\genVTyA}[1]{\text{({$#1$})$\mathcal{I}_d$}}
\newcommand{\genHTyA}[1]{\text{({$#1$})$\mathcal{H}$}}
% mostly unchanged packages that are used in multiple papers
%various packages and definitions used in various documents
%
%
%%%%%%%%%%%%%%%%%%%%%%%%%%%%%%%%%%%%%%%%%%%%%%%%%%%%%%%
%%%%%%%% definitions %%%%%%%
\newtheorem{theorem}{Theorem}[section]
\newtheorem{lemma}[theorem]{Lemma}
%%\newtheorem{corollary}[theorem]{Corollary}
\newenvironment{definition}[1][Definition]{\begin{trivlist}
\item[\hskip \labelsep {\bfseries #1}]}{\end{trivlist}}
%\newenvironment{proof}[1][Proof]{\begin{trivlist}
%\item[\hskip \labelsep {\bfseries #1}]}{\end{trivlist}}
%%
%
\newcommand{\appref}[1]{Appendix~\ref{#1}}
\newcommand{\chapref}[1]{Chapter~\ref{#1}}
\newcommand{\secref}[1]{Section~\ref{#1}}
\newcommand{\tblref}[1]{Table~\ref{#1}}
\newcommand{\figref}[1]{Figure~\ref{#1}}
\newcommand{\eqqref}[1]{Equation~\ref{#1}}
\newcommand{\thmref}[1]{Theorem~\ref{#1}}
\newcommand{\lemref}[1]{Lemma~\ref{#1}}
%\newcommand{\listingref}[1]{Listing~\ref{#1}}
%\newcommand{\pref}[1]{{page~\pageref{#1}}}
\newcommand{\defref}[1]{Definition~\ref{#1} on page \pageref{#1}}
%\newcommand{\ruleref}[1]{Rule~\ref{#1}}
%%
\newcommand{\eg}{{\em e.g.}}
%\newcommand{\cf}{{\em cf.}}
\newcommand{\ie}{{\em i.e.}}
\newcommand{\etc}{{\em etc.\/}}
\newcommand{\naive}{na\"{\i}ve}
%\newcommand{\ala}{{\em \`{a} la\/}}
\newcommand{\etal}{{\em et al.\/}}
\newcommand{\role}{r\^{o}le}
%%\newcommand{\vs}{{\em vs.}}
%%\newcommand{\forte}{{fort\'{e}\/}}
%\newcommand{\Cplusplus}{\mbox{C\hspace{-.05em}
%\raisebox{.4ex}{\tiny\bf ++}}}
%%%%%%%% Colors %%%%%%%
%\definecolor{Red}{rgb}{0.9,0.0,0.0}  % fixme
%\definecolor{Green}{rgb}{0.0,0.4,0.0}
%\definecolor{Blue}{rgb}{0.0,0.0,0.9}
%\definecolor{DarkBlue}{rgb}{0.0,0.0,0.75}
%\definecolor{Midnight}{rgb}{0.0,0.0,0.5}
%\definecolor{Purple}{rgb}{0.5,0.0,0.4}
\definecolor{Black}{rgb}{0.0,0.0,0.0}
%\definecolor{Yellow}{rgb}{1.0,1.0, 0.25}
%\definecolor{Cyan}{rgb}{0.25,1.0, 1.0}
%
\newcommand{\cdColor}{Black}
%\newcommand{\kwColor}{DarkBlue}
%\newcommand{\comColor}{Red}
\newcommand{\mynote}[3]{\textcolor{#3}{\small\textbf{\textsf{{#1}: {#2}}}}}
\newcommand{\CC}[1]{\mynote{cc}{#1}{magenta}}
%
%
%
%%%%%%%% Some common math notation %%%%%%%
%%% double brackets
\newcommand{\LDB}{\ensuremath{[\mskip -3mu [}}
\newcommand{\RDB}{\ensuremath{]\mskip -3mu ]}}
%%
\newcommand{\dom}{\ensuremath{\mathrm{dom}}}
\newcommand{\rng}{\ensuremath{\mathrm{rng}}}
%%
%%% sets
\newcommand{\SET}[1]{\ensuremath{\{#1\}}}
\newcommand{\Fin}{\textrm{Fin}}     % finite power set
\newcommand{\DISJOINT}[2]{\ensuremath{#1 \pitchfork #2}}
\newcommand{\finsubset}{\mathrel{\stackrel{\textrm{fin}}{\subset}}}
%%
%%
%%% finite maps
\newcommand{\finmap}{\mathrel{\stackrel{\textrm{fin}}{\rightarrow}}}
%%\newcommand{\MAP}[2]{\SET{#1 \mapsto #2}}
%%\newcommand{\EXTEND}[2]{\ensuremath{#1{\pm}#2}}
%%\newcommand{\EXTENDone}[3]{\EXTEND{#1}{\MAP{#2}{#3}}}
%%\newcommand{\SUBST}[3]{\ensuremath{#1[#2\mapsto{}#3]}}
%%\newcommand{\SUBSTTWO}[5]{\ensuremath{#1[#2\mapsto{}#3,#4\mapsto{}#5]}}
%%
\definecolor{ngray}{rgb}{0.5,0.5,0.5}

\lstdefinelanguage{Diderot}{%
  % otherkeywords={|,\#},
  morekeywords={%
    bool,%
    die,%
    else,%
    false,field,foreach,%
identity,if,image,load,in,initially,input,int,%
    fem,
    kernel,%
    nan,new,%
    output,%
    real,%
    function,stabilize,strand,string,%
    inside,tensor,true,%
    update,%
    vec2,vec3,vec4,%
    zeros},
  sensitive,%
  morecomment=[s]{/*}{*/},%
  morecomment=[l]//,
  morestring=[b]"}%
  
  
\lstset{
  language=Diderot,
  %basicstyle= \ttfamily,%\rmfamily
  keywordstyle=\bfseries,
  commentstyle=\itshape,
  mathescape=true,
  %numbers=left,
  aboveskip=1.0em, belowskip=1.0em, lineskip=-0.10em,
   numberstyle=\color{ngray}
  }
 %generic input

%%%%%%%%%%%%%%%%%%%%%%%%%%%%%%%%%%%%

%% Use these commands to set biographic information for the title page:
\title{EIN and DATm updates}
\author{Charisee Chiw}


\begin{document}
\maketitle 

\section{Testing}
Once we have added a new operator to the surface language, it is natural to write some test programs by hand. 
The tests we wrote by hand were straightforward, but limited and unhelpful since it missed bugs.
We can apply a more rigorous approach  by using the testing system, 
\checkname{}.
We add the concat operator to \checkname{} (by creating a new operator object)
and used targeted testing to only generate test cases that use the concat operator.
\checkname{} created and ran 126 test programs that use the concat operator.




 \subsection{\checkname{} implementation}
\paragraph{ inside tests}
\paragraph{adjustments}
 \paragraph{ third-arity}
\subsection{Results from using \checkname{}}

\begin{description}
\item[Bug 1] Mistake in index scope when using substitution.
\begin{lstlisting}
field#k(2)[2,2] F0;
field#k(2)[2] F1;
field#k(2)[2] F2;
field#k(2)[2,2] G = (F0 $\circ$ F1) $\bullet$ F2;
\end{lstlisting}

There was a mistake in the substitution method.
The scope of the composition indices were handled incorrectly.
The following is the expected and observed representation of the computation in the \name{} IR.\\
Expected: $e=\sum_{\hat{j}} \expComp{A_{ij}}{B_i}{\beta} *C_j $\\
Observed: $e=\sum_{\hat{k}} \expComp{A_{ik}}{B_i}{\beta} *C_j$\\
in $\lambda(A,B,C)\EinExp{e}{\beta}$(F0,F1,F2). 
\item[Bug 2] Missing cases in split method.\\
Probes of a composition are handled differently before reconstruction.\\
$\sum F(x)$ and $\sum (\text{Comp} (F,G,-))(x)$.\\
Missing case in method leads to a compile time error.
Additionally (Comp(Comp -)-)
\item[Bug 3]  Differentiate a composition\\
The jacobian of a field composition:
\begin{lstlisting}
field#k(d1)[d] F0;
field#k(d)[d1] F1;
field#k(d1)[d,d1] G = $\nabla \otimes$ (F0 $\circ $ F1);
\end{lstlisting}
is represented as $\EinExp{\nabla_j (\text{Comp}(A_i,B_i,i))}{ij}$\\
In accordance with the chain rule ( (f $\circ $ g)' = (f'  $\circ $ g) $\cdot$ g') the rewriting system multiplies the inner derivative (g') with a new composition operation (f'  $\circ $ g).
In practice, the implementation does a point-wise multiplication when it should do an inner product.\\
 Expected: $\sum_{\hat{k}} (\expComp{\nabla_k A_i}{B_i}{\beta} * \nabla_j G_k) $\\
Observed: $\expComp{\nabla_j  A_{\beta}}{B_i}{ i} * (\nabla_j G_i)$\\
in $\lambda(A,B)\EinExp{e}{ij}$(F0,F1). 
\item[False positive]
There was an error in evaluating the correct answer in \checkname{} for multiple applications of the compose operator.
We replace the position variable in one field with the field expression of the other.
The position variables need to be renamed in order to avoid mixing them.
Without renaming them it resulted in a false positive. 

\item[False positive]
\checkname{} generates inside tests to see if we are probing the field in the right position.
They did not account for all the different types of restrictions.



\item[Bug 4]
One was in the creation of the \name{}  operator for the concat operator..

\item[Bug 5]
The bug arose when computing the determinant of the concatenation of a field.
\begin{lstlisting}[mathescape=true]
  field#k(d)[2]F,G;
  field#k(d)[]H = det(concat(F,G));
\end{lstlisting}
The bug was caused by the rewriting system.
Our rewriting system applies index-based rewrites to reduce \name{} expressions. 
A specific index rewrite is applied when the index in the delta term matches an index in tensor (or field)  term.
The rewrite checked if two indices were equal and did not distinguish between variable and constant indices.
It is mathematically incorrect to reduce constant indices, because they are not equivalent to variable indices.
\item[XT6]
Using If Wrapper with other field operators
\begin{lstlisting}[mathescape=true]
field#4(2)[]G = compose(minF((F0),(F1)),(F2*0.1));
\end{lstlisting}
Field operators need to be applied to the leaves in if wrapper. 
The composition is a field operator so it needed to be pushed to the leaves.
$$(\text{If}(c, e3,e4)) \circ es  
\rightarrow
\text{If}(c, e3  \circ es ,e4 \circ es) $$
%:t__p_o32_o26_t1_t13__l2 
\end{description}
\end{document}

